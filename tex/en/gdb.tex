\begin{aosachapter}{GDB}{s:gdb}{Stan Shebs}

GDB, the GNU Debugger, was among the first programs to be written for
the Free Software Foundation, and it has been a staple of GNU systems
ever since.  Originally designed as a plain Unix-type debugger, it has
since been expanded to a wide range of uses, including many embedded
systems, and has grown by an order of magnitude.

This chapter will retrace the development of GDB, showing how the
internal structure derives from new demands, and new features added
over time.

Scope

GDB is designed to be a symbolic debugger for programs written in
compiled imperative languages such as C, C++, Ada, Fortran and so
forth.  As usual for an unaptly-named ``debugger'', it doesn't
actually remove bugs from programs; instead, it helps the developer
study the program's behavior closely, hopefully in enough detail to
find the bugs and figure out how to fix them.  The most common usage
looks something like this:

% gdb myprog
[...]
(gdb) break buggy_function
Breakpoint 1 ...
(gdb) run
Stopped at buggy_function
(gdb) print variable_that_should_be_positive
$$1 = -34
(gdb)

So GDB shows something that is not right, and then it is up to the developer
to figure out why.

The important point for design is that GDB is an interactive toolbox
for poking around in a program, and as such it should be responsive
to an unpredictable series of requests.  It also needs to deal with
programs that have been optimized by the compiler, and programs that
exploit every hardware option for performance, so it needs to have
detailed knowledge down to the lowest levels of a system.

History

GDB is an old program.  It came into existence around 1985-6, written
by Richard Stallman along with GCC, GNU Emacs, and other early
components of GNU.  (In those days, there were no public source
control repositories, and much of the detailed development history
has simply been lost.)

Block Diagram

(big picture)

GDB can be said to have two sides to it:

1) The ``symbol side'' is concerned with symbolic information about
the program.  Symbolic information includes function and variable
names and types, line numbers, machine register usage, and so forth.
The symbol side reads symbolic information from the file, parses
expressions, finds the memory address of a given line number, and so
forth.

2) The ``target side'' is concerned with the manipulation of the
target system.  It has facilities to start and stop the program, to
read memory and registers, to modify them, to catch signals, and so
on.

The two sides are somewhat independent of each other; you can look
around your program's machine code, display variable types, etc,
without having anything running.  Conversely, it is possible to do
pure machine-language debugging even if no symbols are avaiable.

In the middle, tying the two sides together, is the command interpreter
and the main execution control loop.

Symbol Side

The symbol side starts by reading the compiled program and extracting
any symbolic information it finds.

The process starts with BFD, which is the common binary file reading
library shared with the GNU assembler, linker, and other minor tools
(the ``binutils'').  BFD offers a single API to read the original Unix
a.out format, COFF (used on System V Unix and MS Windows), and ELF
(modern Unix, GNU/Linux, and most embedded systems), among others.

BFD just pulls blocks of data from the program into GDB's memory, then
for each format, GDB has two levels of reader.  The first is for basic
symbols, which is just the names that the linker uses to build
executables and library; so these are names with addresses and not
much else.  The second level is detailed symbolic information, which
may be in its own format; for instance, DWARF information is
encapsulated in an ELF file.

The code for reading symbolic information is somewhat tedious and
uninteresting; the different format encode the same information in
their own ways, so it's just a matter of interpretation and construction
of the appropriate GDB structures.

Partial Symbol Tables

However, for a program of significant size (such as Emacs or Firefox),
construction of the symbol table can take quite a while, maybe even
several minutes.  Measurements show that the time is *not* in file
reading, but in the in-memory construction of GDB symbols; there are
literally millions of small interconnected objects, and the time adds
up.

Most of the symbolic information will never be looked at in a session,
since it is local to functions that are not of debugging interest.  So
when GDB first pulls in a program, it scans through the symbolic
information, looking for just the globally-visible symbols and
recording only them in the symbol table.  Complete symbolic info for a
function is filled in only when the user stops inside it.

Partial symbol tables allow GDB to start up in only a few seconds, even
for large programs.

Target Side

Target Vectors and the Target Stack

Originally the target side of GDB was composed of a handful of
platform-specific files that handled the details of calling ptrace(),
launching executables, and so forth.

This is not sufficiently flexible for long-running debugging session,
in which the user might switch from local to remote debugging, switch
from files to live programs, attach and detach, and so forth.

So in 1990 the target side of GDB was redesigned to send all
platform-specific operation through the ''target vector'', which is
basically a class of objects, each of what defines the the specifics
of a type of target.  In practice, each target vector is a structure
of several dozen function pointers.  Methods ranging from reading of
memory and registers, to setting parameters of a tracing run.

In addition, it is often useful to blend methods from several target
vectors.  For instance, consider the printing of an initialized global
variable on Unix; before the program is running, the bytes should come
from the executable's .data section, and in fact there is a target
vector for executable programs for which the method reads from the
file.  But while the program is running, the bytes should instead come
from the process's address space.  So GDB has a ''target stack'' where
the live process target vector gets pushed on top of the executable's
target vector when the process starts running, and is then popped when
it exits.

(picture?)

In practice, the target vector has become a key abstraction in GDB,
while the target stack has not worked so well.  The reality of the
target stack is that target vectors are not really orthogonal to each
other; it never makes sense to have a exec,process,exec,corefile
combination for example, and it would be very confusing if the user
could get into that situation.  So GDB has a notion of ``stratum'' in
which process-like target vectors are all one level, while file-like
target vectors get assigned to a lower level.

While GDBers grumble about the target stack regularly, nobody has yet
come up with (and implemented!) a better replacement.xs

\end{aosachapter}
