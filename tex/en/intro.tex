\begin{aosachapter}{Introduction}{s:intro}{Amy Brown and Greg Wilson}

FIXME

\section*{Contributors}

\emph{Michael Bayer (SQLAlchemy)}: FIXME

\emph{Amy Brown (editorial)}: Amy worked in the software industry for
ten years before decamping to create a freelance editing and book production
business. She has an underused degree in Math, two small children, a
husband and a cat. She can be found online at \url{http://www.arbrown.ca/}

\emph{Jon Cruz (Inkscape)}: FIXME

\emph{Tiago Espinha (Apache Derby)}: FIXME

\emph{Jeff Hardy (Iron Languages)}: FIXME

\emph{Luis Ibanez (ITK)}: Luis has worked for 11 years on the development of
the Insight Toolkit (ITK), an open source library for medical imaging analysis.
Luis is a strong supporter of open access and the revival of reproducibiliy
verification in scientific publishing. Luis has been teaching a course on Open
Source Software Practices at Rensselaer since 2007.

\emph{Mike Kamermans (Processing.js)}: FIXME

\emph{Brad King (ITK)}: FIXME

\emph{Simon Marlow (The Glasgow Haskell Compiler)}: FIXME

\emph{Christoph Otto (Parrot)}: FIXME

\emph{Addy Osmani (jQuery)}: FIXME

\emph{Benjamin Peterson (PyPy)}: FIXME

\emph{Simon Peyton-Jones (The Glasgow Haskell Compiler)}: FIXME

\emph{Stan Shebs (gdb)}: FIXME

\emph{Michael Snoyman (Yesod)}: FIXME

\emph{Jeff Squyres (Open MPI)}: FIXME

\emph{Jesse Vincent (K-9 Mail)}: FIXME

\emph{Barry Warsaw (Mailman)}: FIXME

\emph{Greg Wilson (editorial)}: Greg has worked over the past 25 years
in high-performance scientific computing, data visualization, and
computer security, and is the author or editor of several computing
books (including the 2008 Jolt Award winner \emph{Beautiful Code}) and
two books for children.  Greg received a Ph.D.\ in Computer Science
from the University of Edinburgh in 1993.

\section*{Acknowledgments}

FIXME

\section*{Contributing}

Dozens of volunteers worked hard to create this book, but there is
still lots to do.  You can help by reporting errors, by helping to
translate the content into other languages, or by describing the
architecture of other open source projects.  Please contact us at
\code{aosa@aosabook.org} if you would like to get involved.

\end{aosachapter}
