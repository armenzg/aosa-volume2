\begin{aosachapter}{Git}{s:git}{Susan Potter}

Currently just contains the basic structure of the chapter as I envision 
it today.

Introduction text goes here\ldots

\begin{aosasect1}{Version Control Landscape}

This section will describe the two primary classification dimensions to 
Version Control Systems (VCS) today:

\begin{aosaitemize}
  \item Distribution mode: local, client/server, distributed
  \item Storage mode: changeset based vs directed acyclic graph (DAG) based
\end{aosaitemize}

Provide examples of difference VCSes and how they fit in to these 
classifications, e.g. Subversion (client/server + changeset), Mercurial 
(distributed + changeset), BitKeeper (client/server + DAG), etc.

\end{aosasect1}

\begin{aosasect1}{Gitting Started}

Here I will describe the most rudimentary workflow of installing, configuring 
and cloning a remote Git repository. Then explain on a low-level what happened.

Describe the three main elements of a Git working environment:
\begin{aosaitemize}
  \item Repository (object database): default at #{projectdir}/.git subdirectory
  \item Working area: default at #{projectdir} excluding .git subdirectory
  \item Index: default at #{projectdir}/.git/index file
\end{aosaitemize}

Basic steps:
\begin{aosaitemize}
  \item sudo yum install git-core
  \item git config --global user.name ``Your Name''
  \item git config --global user.email ``user@emaildomain.com''
  \item cat #{HOME}/.gitconfig
  \item git clone remote-repo-url
\end{aosaitemize}

Mention the different protocols that can be used with Git, e.g. ssh, http, 
https, git, file and even rsync. Broadly discuss pros and cons of each.

Protocols notes: 
\begin{aosaitemize}
  \item file:// local repository easiest to setup, usuallly only useful for
    one author environment
  \item ssh:// most common for active development and easy to setup team repository
  \item http:// good for open source pull based (consumers) users, can setup with 
    WebDAV support for allowing pushes, but more work. Good for working in
    restricted public networks where SSH ports may be blocked.
  \item https:// same as http except encrypts pulls and pushes over the wire.
  \item git:// lightweight protocol useful in an enclosed and trusted local network
  \item rsync:// hardly used.
\end{aosaitemize}

\end{aosasect1}

\begin{aosasect1}{Gitting Down \& Dirty}

Start with simplest of workflows: purely LOCAL. One author/editor local repo,
multiple contributors receive bundle of their branch from author/editor.
Contributors submit patches to author/editor (via email or ticketing system):
1. Contributor: git format-patch origin/master --stdout > my-feature.patch
2. Author: git am < my-feature.patch

Notice how GIT\_AUTHOR\_NAME and GIT\_AUTHOR\_EMAIL use the contributors
values and the editor's information is in GIT\_COMMITTER\_NAME, etc.

Make sure to cover: git config|init|add|commit|status|branch|tag|log|
format-patch|am|merge|rebase|bundle|diff

% This is how to include a figure using the AOSA styles already configured
%\aosafigure{../images/git/distworkflow.eps}{Distributed VCS}{fig.git.dvcs}

Discussion on distributed workflows\ldots

Introduce Git Flow workflow as a common one used in the Git community. Provide
a realistic example. Plus I need to create good diagrams to explain things
better.

Then introduce the ``fork'' workflow which is commonly seen on GitHub and
used by OSS projects.

Then introduce the tiered/curated workflow where a release engineer/manager
is gatekeeper. production-like environments do not have development branches
visible to them at all. More locked-down.

Then introduce the GitHub model with ``forks''.

\end{aosasect1}

\begin{aosasect1}{Deploying with Git}

Here talk a little more about 'git bundle' and git shallow clones, e.g.
git clone --depth N #{GIT\_URL}, where N is the number of parents.

Demonstrate with a repository that shows significant benefit to using
bundle in conjunction with shallow clones as a way to provide portable
shall git repositories that allow deployment agents to transfer/receive
minimal bytes over the wires AND have ability to do rollbacks in severe
cases where a release does not go as planned.

\end{aosasect1}

\end{aosachapter}
