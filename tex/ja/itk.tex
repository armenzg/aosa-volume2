\begin{aosachapter}{ITK}{s:itk}{Luis Ibanez and Brad King}
%% Based on EN-Revision r324

The Insight Toolkit (ITK) is a library for image analysis

% Your chapter goes here---please look at /volume1/tex/en/wesnoth.tex for 
% formatting ideas.

\begin{aosasect1}{What Is ITK?}
\end{aosasect1}

\begin{aosasect1}{Architectural Features}

\begin{aosasect2}{Pipeline Architecture}
\end{aosasect2}

\begin{aosasect2}{Factories}
\end{aosasect2}

\begin{aosasect2}{IO Factories}
\end{aosasect2}

\begin{aosasect2}{Streaming}
\end{aosasect2}

\end{aosasect1}

\begin{aosasect1}{On Form and Function}

The Architecture of ITK was not solely the outcome of techincal
considerations. It is the result of a process that lives in the
context of social interactions, driving applications, and
institutional missions. To fully understand the architecture, it is
important to put it in the context of that larger community from where
the architectural decisions arouse. In the following section we
describe some of the non-technical driving forces that shaped the
architecture of the software In the following section we describe some
of the non-technical driving forces that shaped the architecture of
the software.

\begin{aosasect2}{Shaped by a Community}
There is no software without humans around it. A software is only
useful when surrounded by a community that takes care of it. The
community is indeed more important that the software itself, to the
point that the software should be seen simply as the glue that keeps
that community together and as the platform where the \emph{``collective
intelligence''} of the community gets imprinted.
\end{aosasect2}

\begin{aosasect2}{Shaped for Maintenance}
The architecture satisfies the constraints that minimize maintenace cost.
\begin{itemize}
\item Modularity (at the class level)
\item Many small files
\item Code reuse
\item Repeated patterns
\end{itemize}
\end{aosasect2}
As the developers got involved in regular maintenance activities, they
got exposed to the ``common failures'' of certain details. The things
that raised common questions in the mailing lists, the details that
new developers tend to miss and that led them to introduce buggy code.
After dealing with such issues, developers learned to write code that
is ``good for maintenance''. Some of this traits apply to both coding
style and the actual organization of the code.

\begin{aosasect2}{The Invisible Hand}
The software should look like writted by a single person. The best
developers are the ones that write code that can be taken over by
anybody else, should they be taken down by the ``Provervial Bus'' when
crossing a street. We have grown to recognize that any trace of
``personal touch'' is an indication of a defect introduced in the
software.
\end{aosasect2}

\end{aosasect1}

\begin{aosasect1}{Refactoring}
ITK Started in the year 2000 and grew continuously until the year
2011. The development team had the truly unique opportunity to embark
in a refactoring effort under the funding of the National Library of
Medicine. This is not a minor feat. Once you have been working on a
piece of software for over a decade, and you are offered the
opportunity to clean it up: What would you change ?.
\end{aosasect1}

\begin{aosasect1}{Reproducible Research}
One of the early lessons learned in ITK was that the many papers
published in the field were not as easy to implement as we were led to
believe. The computational field tend to over-celebrate algorithms and
to dismiss the practical work of writing software as ``just an
implementation detail''. That dismissive attitude is quite damaging to
the field, since diminshes the importance of the first-hand experience
with the code and its proper use. The outcome is that most published papers
are not reproducible, and that when researchers and students attempt to use
such techniques, they end up spending a lot of time in the process and deliver
variations of the original work. It is actually quite difficult in practice
to verify if an implementation matches what was described in a paper. 

ITK disrupted, for the good, that environment and restored a culture
of DYI (Do It Yourself), in a field that has grown accoustomed to
theoretical reasoning, and that had learned to dismiss experimental
work. The new culture brought by ITK is a practical and pragmatic one 
in which the virtues of the software are judged by its practical results
and not by the appearance of complexity that is celebrated in scientific
publications. Helas, it turns out that in practice, the most effective 
processing methods are those that would appear to simple to be accepted 
for a scientific paper.

The Insight Journal...
\end{aosasect1}

\end{aosachapter}
